\documentclass{article}

% Packages for formatting
\usepackage[margin=1in]{geometry}
\usepackage{fancyhdr}
\usepackage{amssymb}
\usepackage[most]{tcolorbox}
\usepackage{listings}
\usepackage{siunitx}
\usepackage{cite}


\newtcblisting{mycode}{%
    boxrule=0.5pt,
    colback=white,
    listing only,
    listing options={basicstyle=\ttfamily, numbers=left},
}
\lstset{basicstyle=\ttfamily}
\pagestyle{fancy}
\lhead{Simon Lavoie}
\chead{}
\rhead{August 18, 2024}
\lfoot{}
\cfoot{}
\rfoot{}
\renewcommand{\headrulewidth}{1pt}

\date{October 12, 2023} % Leave this blank to remove the date from the title

\begin{document}


\large
\begin{center}
\section*{The Search for Multiquark States at the Large Hardon Collider}
\end{center}
\normalsize 




\section*{Abstract}
Baryonic matter, of which we are primarily composed of, is itself composed of three quarks.
These baryons include the proton and neutron, which we know as the core constituents of the 
nucleus. There also exists the traditional mesons, which are composed of only two quarks. This begs the question:
do particles composed of groups larger than a trio of quarks exist? The short answer is yes, 
with the first observation of a tetraquark - a particle composed of four quarks - being made 
in 2003 by the Belle Collaboration \cite{MultiquarkDiscovery}.
In this paper, the search for possible multiquark states using strange particles, namely, the 
$K^0_s$ meson and the $\Lambda^0$ baryon is done using data collected by the ATLAS detector 
of proton-proton collisions at a center-of-mass energy of $\sqrt{s} =$ \SI{13.6}{TeV}. Minimum
bias data from 2015 boasting an integrated luminosity of \SI{21.559}{nb}$^{-1}$ is used. Secondary 
vertices are marked as either a $K^0_s$ or $\Lambda^0$ using strict cuts and their invariant mass and 
lifetimes are fitted for verification. A low energy resonant state of the $K^0_sK^0_s$ invariant 
mass distribution was detected at 1524.40 $\pm$ 4.83 MeV which likely corresponds to the 
$f_2'(1525)$ state. With this known signal identified, a bump hunt is performed on the higher-energy
range of the $K^0_sK^0_s$, $K^0_s\Lambda^0$ and $\Lambda^0\Lambda^0$ invariant mass spectra
in search for possible tetraquark, pentaquark and hexaquark states. DID I FIND ANYTHING?

\section{Introduction}
From the discovery of various exotic particles such as pions in the 50's to that of the Higgs
boson in 2012, the advent of particle accelerators brought about new discoveries in particle physics,
bolstering our grasp on the standard model - a guage theory which describes three out of the four
fundamental forces, that being the weak, strong, and electromagnetic forces \cite{KLDiscovery}, \cite{HiggsDiscovery}. 
As the collision energies of particle accelerators steadily increase, new discoveries of higher mass particles 
can be made. 
Multiquarks are an example of these higher mass particles, being made up of collections of quarks
whose number exceeds that of the usual two or three we observe in mesons and baryons respectively.
Such particles are not forbidden by the standard model, and a proper identification of multiquark 
states may help in our understanding of quantum chromodynamics (QCD) - a core component of the standard 
model which describes the interactions between quarks and the elementary particle which mediates 
the strong interaction between them: the gluon. Thus, this work searches for multiquark states 
through the possible decay products of tetraquarks, pentaquarks and hexaquarks. The decay channels
are as follows:

\begin{align*}
\text{Tetraquark}: X &\rightarrow K^0_sK^0_s \\
\text{Pentaquark}: Y &\rightarrow K^0_s\Lambda^0 \\
\text{Hexaquark}:  Z &\rightarrow \Lambda^0\Lambda^0
\end{align*}

Where the $K^0_s$ meson is a neutral kaon composed of a down quark $(d)$ and an anti-strange quark 
$(\bar{s})$, and the $\Lambda^0$, known as the Lambda baryon, is composed of an up, down and strange 
quark $(uds)$. Quarks are known to possess one of the three fundamental color charges, of which there 
are three: Red, Green or Blue. According to the color confinement principle, a core tenet of QCD,
color-charged particles such as quarks cannot be isolated, and thus quarks combine to form color-neutral 
hadrons called singlet states. Nothing forbids the creation of a singlet state composed of two 
quarks and two antiquarks $(qq\bar{q}\bar{q})$, a particle called a tetraquark which would extend 
the definition of a traditional meson from a single quark-antiquark pair to any particle composed 
in equal parts of quarks and antiquarks. Similarly, nothing forbids the combination of a color-neutral 
baryon and a quark-antiquark pair $(qqqq\bar{q})$ which we would call a pentaquark. A combination of 
two baryons or three quark-antiquark pairs could also exist $(q^6 \text{ or } q^3\bar{q}^{3})$, 
which are both example of hexaquarks.

\subsection{Event Reconstruction}
The $K^0_s$ and $\Lambda^0$ are chosen as objects of study due to their decay products being easily 
identifiable within the ATLAS detector. The $K^0_s$ is known to decay into two charged pions $(\pi^+\pi^-)$
with a branching ratio of $69.20 \pm 0.05$ \unit{\%} \cite{KBranchingRatio} 

\bibliographystyle{plain}  
\bibliography{bibliography} 

\end{document}
